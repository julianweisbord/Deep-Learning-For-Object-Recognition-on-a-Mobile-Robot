\documentclass[draftclsnofoot, onecolumn, 10pt, compsoc]{IEEEtran}
%% Language and font encodings
\usepackage[english]{babel}
\usepackage{amsmath}
\usepackage{graphicx}
\usepackage[top=0.75in, bottom=0.75in, left=0.75in, right=0.75in]{geometry}

%% Macros
\newcommand\tab[1][1cm]{\hspace*{#1}}

%% Useful packages
\usepackage{url}
\usepackage{pgfgantt}
\usepackage{comment}

\title{Group 67 - Technology Review}

\author{
	Deep Sequential Learning for Object Recognition on a Mobile Robot \\
	Miles McCall\\
	Oregon State University \\
	CS 461 Fall 2017
}

\begin{document}
	\maketitle
	
	\begin{abstract}
		\noindent The Technology Review is used to break the project down into its core components with three main points describing each section. For each point, three technology options are compared and contrasted, and the one most likely to be implemented for the project is explained. 
	\end{abstract}
	
	\tableofcontents
	\newpage
	
	\section{Role}
		Our group has three members, and we all have equal responsibility across the project. At this point we are all working on every piece of the project to establish a consistent knowledge base. Further along into developing our pipeline we may divide tasks more and gain more expertise in a specific portion of the project, but for now we are all working together equally. 
		
		I am particularly interested in the image classification and neural network involved in the project. While the whole project is technically about deep learning and object recognition, the portion that covers the details, implementation, and design philosophy behind these sections is what I would prefer to focus on. 
	
	\section{What You Are Trying to Accomplish}
		Our group is working to rewrite and improve upon the current software on the Fetch robot at the Personal Robotics Lab on campus. Through this redesign we aim to recreate the data collection and image recognition systems currently in place.
		
		While the data gathering software currently being used on the robot, or intelligent agent, is likely adequate in its current state, our group would like an improved version to match the new classification design as well. By adding tag sensing to the current object location procedure, we hope to make image localization more accurate. The intelligent agent will be able to scan its environment searching for the tags placed on each test object. With a better measurement of where the object exists in 3D space, the intelligent agent will better center itself around the object and take more on-target images. This will improve the quality of our input and training data sets, and allow the classification pipeline to learn and process on more accurate representations of the objects. 
		
		Part of the way we plan to improve the image classifier is by implementing a Sequential Online Learning neural network model, as opposed to the currently running Batch Learning model. The goal is for the robot to be capable of training itself with an initial dataset, then gathering further data as needed to make the CNN more accurate. By using a more complicated model we should be able to append to the created datasets the robot already recognizes. 
		
		I plan to improve and build upon my own skills and knowledge base throughout the course of the project. While we will all be contributing to every part of the design process, I wish to become especially competent in the implementation of our full classification pipeline. I'd like to accomplish a full hierarchy of neural networks with sequential learning implemented over the top of the classifiers. 
	
	\section{Tech Review}
		In this technology review I will be focusing on the first half of our image classification model, covering the overarching classifiers, sub-classifiers, and overfitting in the algorithm. For each subject I will analyze three potential options for the technology and their different trade-offs. 
		
		\subsection{Project Overview}
			\begin{itemize}
				\item Data Collection
					\begin{itemize}
						\item ROS for robot actuation
						\item Image Tagging - Object Selection with Interface
						\item Data Gathering
					\end{itemize}
				
				\item \textbf{Image Recognition Part 1}
					\begin{itemize}
						\item \textbf{Overarching Classifiers}
						\item \textbf{Sub Classifiers}
						\item \textbf{Overfitting in the Algorithms}
					\end{itemize}
				
				\item Image Recognition Part 2
					\begin{itemize}
						\item Backpropagation Methods
						\item Types of Online Learning
						\item Minimizing Catastrophic Interference
					\end{itemize}
			\end{itemize}
		
		\subsection{Overarching Classifier}
		
		\subsection{Sub Classifiers}
		
		\subsection{Overfitting in the Algorithms}

		\begin{comment}
			Three possible technologies that could be used to accomplish the different pieces you selected to examine in more detail. Identify these potential technologies even if your client has told you which one to use.
			After conducting research and analyzing trade-offs, identify which technology you have selected for each piece of the project, and why. Convince the reader your analysis is unbiased and well-considered.
		\end{comment}
	
	%\newpage
	%\bibliographystyle{IEEEtran}
	%bibliography{tech_review_current}
\end{document}